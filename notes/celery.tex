\documentclass [8pt] {extarticle}

\usepackage {geometry}
\usepackage {multicol}

\geometry {
    a4paper,
    lmargin = 20mm,
    tmargin = 10mm,
    rmargin = 10mm,
    bmargin = 20mm,
}

\parindent=0mm

\pagestyle {empty}

\setcounter {secnumdepth} {0}

\usepackage {enumitem}
\setlist {nosep}

\usepackage {titlesec}
\titleformat \subsection
    {\hyphenpenalty=10000 \bf} {\thesubsection~} {0mm} {}


\begin {document}
    \section {Celery Contents}

    \begin {multicols} {2}

    \begin {itemize}
\item Copyright

\item Getting Started
    \begin {itemize}

\item Introduction to Celery

    \item Backends and Brokers
    \item First Steps with Celery
    \item Next Steps
    \item Resources
    \end {itemize}
    \item User Guide
        \begin {itemize}

\item Application

    \begin {itemize}
        \item Main Name
        \item Configuration
        \item Laziness
        \item Breaking the chain
        \item Abstract Tasks
    \end {itemize}

\item Tasks

    \begin {itemize}
        \item Basics
        \item Names
        \item Task Request
        \item Logging
        \item Retrying
        \item List of Options
        \item States
        \item Semipredicates
        \item Custom task classes
        \item How it works
        \item Tips and Best Practices
        \item Performance and Strategies
        \item Example
    \end {itemize}

\item Calling Tasks

    \begin {itemize}
        \item Basics
        \item Linking (callbacks/errbacks)
        \item On message
        \item ETA and Countdown
        \item Expiration
        \item Message Sending Retry
        \item Connection Error Handling
        \item Serializers
        \item Compression
        \item Connections
        \item Routing options
        \item Results options
    \end {itemize}

\item Canvas: Designing Work-flows

    \begin {itemize}
\item Signatures (Partials, Immutability, Callbacks)
\item The Primitives (Chains, Groups, Chords, Map \& Starmap, Chunks)
    \end {itemize}

\item Workers Guide

    \begin {itemize}
        \item Starting the worker
        \item Stopping the worker
        \item Restarting the worker
        \item Process Signals
        \item Variables in file paths
        \item Concurrency
        \item Remote control
        \item Commands
        \item Time Limits
        \item Rate Limits
        \item Max tasks per child setting
        \item Max memory per child setting
        \item Autoscaling
        \item Queues
        \item Inspecting workers
        \item Additional Commands
        \item Writing your own remote control commands
    \end {itemize}

\item Daemonization

    \begin {itemize}
        \item Generic init-scripts (celeryd, celerybeat)
        \item Usage systemd
        \item Running the worker with superuser privileges (root)
    \end {itemize}

\item Periodic Tasks

    \begin {itemize}
\item Introduction
\item Time Zones
\item Entries
\item \quad    Available Fields
\item Crontab schedules
\item Solar schedules
\item Starting the Scheduler
\item \quad   Using custom scheduler classes
    \end {itemize}

\item Routing Tasks

    \begin {itemize}
        \item Basics
        \item Special Routing Options
        \item AMQP Primer
        \item Routing Tasks
    \end {itemize}

\item Monitoring and Management Guide

    \begin {itemize}
\item Workers (CLI, Flower - web-monitor, events - curses monitor)
\item RabbitMQ
\item Redis
\item Munin
\item Events
\item Event Reference
    \end {itemize}

\item Security

    \begin {itemize}
\item Areas of Concern (Broker, Client, Worker)
\item Serializers
\item Message Signing
\item Intrusion Detection (Logs, Tripwire)
    \end {itemize}

\item Optimizing
\item Debugging
\item Concurrency
\item Signals

    \begin {itemize}
        \item Task Signals
        \item App Signals
        \item Worker Signals
        \item Beat Signals
        \item Eventlet Signals
        \item Logging Signals
        \item Command signals
        \item Deprecated Signals
    \end {itemize}

\item Testing with Celery
\item Extensions and Bootsteps

    \begin {itemize}
        \item Custom Message Consumers
        \item Blueprints
        \item Worker
        \item Consumer
        \item Installing Bootsteps
        \item Command-line programs
        \item Worker API
    \end {itemize}

\item Configuration and defaults

    \begin {itemize}
\item Example configuration file
\item New lowercase settings
\item Configuration Directives
    \begin {itemize}
        \item General settings
        \item Time and date settings
        \item Task settings
        \item Task execution settings
        \item Task result backend settings
        \item Backend settings: \\
        Database,
        RPC,
        Cache,
        MongoDB,
        Redis,
        Cassandra,
        S3,
        Azure Block Blob,
        Elasticsearch,
        AWS DynamoDB,
        IronCache,
        Couchbase,
        ArangoDB,
        CosmosDB,
        CouchDB,
        File-system,
        Consul K/V store
        \item Message Routing
        \item Broker Settings
        \item Worker
        \item Events
        \item Remote Control Commands
        \item Logging
        \item Security
        \item Custom Component Classes (advanced)
        \item Beat Settings (celery beat)
    \end {itemize}
    \end {itemize}

\item Documenting Tasks with Sphinx
        \end {itemize}

    \item Django
    \item Contributing
    \item Community Resources
    \item Tutorials (Ensuring a task is only executed one at a time)
    \item Frequently Asked Questions
    \item Change history
    \item What’s new in Celery 5.2 (Dawn Chorus)
    \item API Reference
    \item Internals
    \item History
    \item Glossary
    \end {itemize}

    \end {multicols}

    \pagebreak

    \section {Celery Glossary}

\textbf {ack} \quad Short for acknowledged. \\
\textbf {acknowledged} \quad Workers acknowledge messages to signify that a
message has been handled. Failing to acknowledge a message will cause the
message to be redelivered. Exactly when a transaction is considered a failure
varies by transport. In AMQP the transaction fails when the connection/channel
is closed (or lost), but in Redis/SQS the transaction times out after a
configurable amount of time (the visibility\_timeout). \\
\textbf {apply} \quad Originally a synonym to call but used to signify that a
function is executed by the current process. \\
\textbf {billiard} \quad Fork of the Python multiprocessing library containing
improvements required by Celery. \\
\textbf {calling} \quad Sends a task message so that the task function is
executed by a worker. \\
\textbf {context} \quad The context of a task contains information like the id
of the task, it’s arguments and what queue it was delivered to. It can be
accessed as the tasks request attribute. See Task Request \\
\textbf {early ack} \quad Short for early acknowledgment \\
\textbf {early acknowledgment} \quad Task is acknowledged just-in-time before
being executed, meaning the task won’t be redelivered to another worker if the
machine loses power, or the worker instance is abruptly killed, mid-execution.
Configured using task\_acks\_late. \\
\textbf {ETA} \quad “Estimated Time of Arrival”, in Celery and Google Task Queue,
etc., used as the term for a delayed message that should not be processed
until the specified ETA time. See ETA and Countdown.  \\
\textbf {executing} \quad Workers execute task requests.  \\
\textbf {idempotent} \quad Idempotence is a mathematical property that describes a
function that can be called multiple times without changing the result.
Practically it means that a function can be repeated many times without
unintended effects, but not necessarily side-effect free in the pure sense
(compare to nullipotent). \\
 \\
\textbf {kombu}    Python messaging library used by Celery to send and receive
messages. \\
\textbf {late ack}    Short for late acknowledgment \\
\textbf {late acknowledgment}    Task is acknowledged after execution (both if
successful, or if the task is raising an error), which means the task will be
redelivered to another worker in the event of the machine losing power, or the
worker instance being killed mid-execution. Configured using task\_acks\_late.
\\
\textbf {nullipotent} \quad describes a function that’ll have the same effect,
and give the same result, even if called zero or multiple times (side-effect
free). A stronger version of idempotent. \\
\textbf {pidbox} \quad A process mailbox, used to implement remote control
commands. \\
\textbf {prefetch count} \quad Maximum number of unacknowledged messages a
consumer can hold and if exceeded the transport shouldn’t deliver any more
messages to that consumer. See Prefetch Limits. \\
\textbf {prefetch multiplier} \quad The prefetch count is configured by using
the worker\_prefetch\_multiplier setting, which is multiplied by the number of
pool slots (threads/processes/greenthreads). \\
\textbf {reentrant} \quad describes a function that can be interrupted in the
middle of execution (e.g., by hardware interrupt or signal), and then safely
called again later. Reentrancy isn’t the same as idempotence as the return
value doesn’t have to be the same given the same inputs, and a reentrant
function may have side effects as long as it can be interrupted; An idempotent
function is always reentrant, but the reverse may not be true. \\
\textbf {request} \quad Task messages are converted to requests within the
worker. The request information is also available as the task’s context (the
task.request attribute). \\

\section {Celery FAQ}
    \begin {multicols} {2}

    \subsection {General}

        What kinds of things should I use Celery for?

    \subsection {Troubleshooting}

        MySQL is throwing deadlock errors, what can I do?

        The worker isn’t doing anything, just hanging

        Task results aren’t reliably returning

        Why is Task.delay/apply*/the worker just hanging?

        Does it work on FreeBSD?

        I’m having IntegrityError: Duplicate Key errors. Why?

        Why aren’t my tasks processed?

        Why won’t my Task run?

        Why won’t my periodic task run?

        How do I purge all waiting tasks?

        I’ve purged messages, but there are still messages left in the queue?

    \subsection {Results}

        How do I get the result of a task if I have the ID that points there?

    \subsection {Security}

        Isn’t using pickle a security concern?

        Can messages be encrypted?

        Is it safe to run celery worker as root?

    \subsection {Brokers}

        Why is RabbitMQ crashing?

        Can I use Celery with ActiveMQ/STOMP?

        What features aren’t supported when not using an AMQP broker?

    \subsection {Tasks}

        How can I reuse the same connection when calling tasks?

        sudo in a subprocess returns None

        Why do workers delete tasks from the queue if they’re unable to process them?

        Can I call a task by name?

        Can I get the task id of the current task?

        Can I specify a custom task\_id?

        Can I use decorators with tasks?

        Can I use natural task ids?

        Can I run a task once another task has finished?

        Can I cancel the execution of a task?

        Why aren’t my remote control commands received by all workers?

        Can I send some tasks to only some servers?

        Can I disable prefetching of tasks?

        Can I change the interval of a periodic task at runtime?

        Does Celery support task priorities?

        Should I use retry or acks\_late?

        Can I schedule tasks to execute at a specific time?

        Can I safely shut down the worker?

        Can I run the worker in the background on [platform]?

    \subsection {Django}

        What purpose does the database tables created by django-celery-beat have?

        What purpose does the database tables created by django-celery-results have?
        \end {multicols}

\end {document}
